\documentclass[12pt]{article}

% For formatting
\usepackage[margin=1truein]{geometry}
\usepackage{fancyhdr}
\usepackage{setspace}
\usepackage{titlesec}
\usepackage{indentfirst}

\usepackage[style=apa, backend=biber]{biblatex}
\addbibresource{refs.bib}

% Font
\usepackage{newtxtext}

% For starting the page number on the title page
\usepackage{etoolbox}
\patchcmd{\titlepage}
    {\thispagestyle{empty}}
    {\thispagestyle{fancy}}
    {}
    {}

\titleformat*{\section}{\centering\normalsize\bfseries}
\titlespacing*{\section}{0pt}{0pt}{0pt}
\titleformat*{\subsection}{\normalsize\bfseries}
\titlespacing*{\subsection}{0pt}{0pt}{0pt}

\pagestyle{fancy}
\fancyhf{}
\fancyhead[R]{\thepage}
\renewcommand{\headrulewidth}{0pt}
\setlength{\headheight}{15pt}

\onehalfspacing
\begin{document}
    \begin{titlepage}
  \begin{center}
    {\doublespacing
      \vspace*{4\baselineskip}

      \textbf{\mytitle}
      \vspace*{\baselineskip}

      Jaden Nola

      Department of English, Wichita State University

      LING 151: Nature of Language

      Kaitlyn Hemberger

      May 4, 2023\par
    }
  \end{center}
\end{titlepage}

    \setcounter{page}{2}
    {\centering
      \textbf{An Application of Grice's Maxims of Conversation to Ownership 
        in Rust}\par
    }
    Why are programming languages not considered bonafide language? Although
    they resemble human language, they lack certain design features that we
    usually find in natural language. However, the extent to which programming
    languages resemble natural language is an interesting discussion. This
    paper will attempt to apply Grice's maxims of conversation to the concept
    of ownership in the Rust programming language.
    \section*{Background}
    \subsection*{Grice's Maxims of Conversation}
    According to \textcite[pp.\@~280-283]{dawson_phelan_2016}, H. P. Grice
    described maxims of conversation that kept discourse coherent, maxims
    which were put into four categories. The \textit{maxims of quality} are
    the expectation that participants in a conversation provide, at least 
    according to their current knowledge, truthful information. The 
    \textit{maxims of relevance} describe that utterances in conversation 
    should be relevant, as the name implies. The \textit{maxims of quantity}
    dictate that speakers should not give too little nor too much
    information. Finally, the \textit{maxims of manner} require that
    information is given in a direct and easily understood manner.
    \subsection*{Ownership in Rust}
    Memory management poses a difficult challenge for creators of programming
    languages, as this process may influence future idiomatic practices or
    affect the performance of the language. Rust is a relatively new
    programming language which boasts more effective memory safety due to its
    ownership model, a unique memory management 
    process~\autocite{rust_website}.  This model is most aptly explained by an
    analogy.
    
    Consider the following sentence: "My dog, \textit{who is the same breed}
    \textit{as the dog we saw chasing the squirrel}, drank water this
    morning." We have three entities in this sentence: the speaker's dog,
    the squirrel, and the dog chasing the squirrel; however, the squirrel and
    the dog chasing the squirrel only exist within the dependent clause while
    the speaker's dog exists throughout the entire phrase. For the sake of
    analogy, we will say that the squirrel and the squirrel-chasing dog 
    \textit{belong} to the dependent clause. Data works similarly in Rust,
    where they no longer exist after their \textit{scope}~(dependent~clause)
    is finished executing. In this system, variables can give ownership of
    their data to variables in other scopes, keeping the data from being
    lost.

    Here we will also briefly define references, as they pertain to memory
    management. While variables can hold values, they can also hold memory
    addresses called \textit{references}, which allow indirect access to the
    value. This fits into the ownership model as other scopes can be given
    references to a variable instead of the variable itself. When those scopes
    end, only the reference will be thrown out and not the value. It is
    easily understood that references cannot exist when the value it
    references has been thrown out. We should also note that references
    introduce a problem: simultaneous access to the same value. This could
    result in one scope j
    \section*{Research Question}
    How do ownership rules in Rust resemble or enforce Grice's maxims of
    conversation?
    \section*{Methodology}
    Compiled from \textcite{rust_ownership} and \textcite{rust_references}
    \begin{itemize}
        \item A variable cannot write to its data if there is a reference
            pointing to the data
        \item A variable cannot be accessed if there is a mutable reference
            pointing to the data
        \item Data must outlive all of its references
        \item Do not make a reference unnecessarily mutable
        \item 
    \end{itemize}
    \newpage
    \printbibliography
\end{document}
