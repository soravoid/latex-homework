\documentclass[12pt, titlepage]{article}

% For formatting
\usepackage[margin=1truein]{geometry}
\usepackage{fancyhdr}
\usepackage{setspace}
\usepackage{titlesec}
\usepackage{graphicx}
\usepackage{float}
\usepackage{color}
\usepackage{listings}
\usepackage{cite}

% For equations
\usepackage{amsmath, amssymb}
\usepackage{physics}
\usepackage{newtxtext, newtxmath}

\newcommand{\mytitle}{Numerical Solutions to Elliptical Partial Differential Equations}

\definecolor{verbgray}{gray}{0.9}

% New code environment that is just the verbatim environment with color bg
\lstnewenvironment{code}{
    \lstset{backgroundcolor=\color{verbgray},
    frame=single,
    framerule=0pt,
    basicstyle=\ttfamily,
    columns=fullflexible}
}{}

\definecolor{shadecolor}{rgb}{.9, .9, .9}

\title{\mytitle}
\author{Jaden Nola}
\date{2 May 2023}

\pagestyle{fancy}
\fancyhf{}
\fancyhead[C]{\mytitle}
\fancyhead[R]{\thepage}
\renewcommand{\headrulewidth}{0pt}
\setlength{\headheight}{15pt}

\setlength{\parskip}{\baselineskip}
\titleformat{\section}{\normalfont\fontsize{15.6}{15.6}\bfseries}{\thesection}{1em}{}
\titlespacing{\section}{0pt}{\parskip}{-\parskip}
\titleformat{\subsection}{\normalfont\fontsize{13.2}{13.2}\bfseries}{\thesubsection}{1em}{}
\titlespacing{\subsection}{0pt}{\parskip}{-\parskip}

\doublespacing
\begin{document}
    \maketitle
    \section{Background}
    According to Greenbaum~\&~Chartier~\cite{greenbaum}, a \textbf{partial differential equation} (PDE) is an expression which can be written
    in the following form:
    \begin{equation}\label{eq:pde}
        au_{xx} + 2bu_{xy} + cu_{yy} + du_{x} + eu_{y} + fu = g
    \end{equation}
    where \textit{a}, \textit{b}, \textit{c}, \textit{d}, \textit{e}, \textit{f}, and \textit{g} are functions of \textit{x} and \textit{y}.
    From this form, we are able to further classify different types of PDEs based on a quantity known as
    the \textbf{discriminant}. It is defined as
    \begin{equation}
        D \equiv b^2 - ac
    \end{equation}
    When $D < 0$, we say that the equation is \textbf{elliptical}; if $D > 0$, then it is \textbf{hyperbolic}; 
    otherwise, if $D = 0$, then the equation is \textbf{parabolic}. As \textit{a}, \textit{b}, and \textit{c} are functions of \textit{x} and \textit{y},
    the type of PDE represented by Equation~(\ref{eq:pde}) can change types throughout the domain of the solution function.
    This paper will focus on numerical solutions to differential equations to elliptical PDEs.  
    \section{Problem Statement}
    In this paper, we will specifically focus on the \textbf{Poisson~equation}~\cite{burden_faires_2011}, which is given by
    \begin{equation}\label{eq:poisson}
        \grad^2 u(x,y) \equiv \pdv[2]{u}{x}~(x,y) + \pdv[2]{u}{y}~(x,y) = f(x,y)
    \end{equation}
    where $\grad = \text{grad}f = \hat{\imath}\pdv{x} + \hat{\jmath}\pdv{y}$ and $\grad^2 = \grad \cdot \grad = \hat{\imath}\pdv[2]{x} + \hat{\jmath}\pdv[2]{y}$.
    In the case that Equation~(\ref{eq:poisson}) is homogenous (i.e., $f(x,y)=0$), then
    \begin{equation}\label{eq:laplace}
        \grad^2 u(x,y) = 0
    \end{equation}
    and Equation~(\ref{eq:laplace}) is called the \textbf{Laplace~equation}. Observe that since $b = 0$ and $a = c = 1$, 
    the discriminant of Equation~(\ref{eq:poisson}) is $D=-1$, thus the Poisson equation is always elliptic in any domain.

    Our goal is to approximate the solution to this PDE on the open rectangle 
    \begin{equation*}
        R = \left\{(x,y)\; |\; a < x < b,\; c < y < d\right\} \subset \mathbb{R}^2
    \end{equation*}
    Furthermore, we are also subject to boundary conditions $u(x,y) = g(x,y)\;\forall\:(x,y)\in S$, where \textit{S} is
    the boundary of the rectangle \textit{R}. If we have that \textit{f} and \textit{g} are continuous on \textit{R}, then 
    we are guaranteed a unique solution to Equation~(\ref{eq:poisson}).
    \section{Method}
    This section will outline the method described in~\cite{burden_faires_2011}. Similar to our methods of approximating linear ordinary differential equations, we will split our rectangle into
    $n+1$ subintervals along the \textit{x}-direction and $m+1$ subintervals of equal length along the \textit{y}-direction, where $m,n \in \mathbb{N}$.
    Thus, we define $h=(b-a)/n$ and $k=(d-c)/m$, which gives $x_i = a  + ih,\; i=0,1,\dots,n$ and $y_j = c + kj, \; k=0,1,\dots,m$. These $(n+1)(m+1)$ points
    form the border and intersections of our \textbf{grid}. In our grid, there are $(n-1)(m-1)$ points that do not lie on the border, which we will call \textbf{mesh points}.

    We assume that our solution function $u(x,y) \in C^4~(R)$, meaning that the fourth derivative is continuous on \textit{R}. In order to find an appropriate approximation to Equation~(\ref{eq:poisson}),
    we apply Taylor's theorem in the \textit{x}-direction. Carrying out this procedure around each mesh point $x_i$,\\
    $i=1,2,\dots,n-1$, we find
    \small
    \begin{align*}
        u(x_i + h, y_j) &= u(x_i, y_j) + \pdv{u}{x}~(x_i,y_j)h+\frac{1}{2!}\pdv[2]{u}{x}~(x_i,y_j)h^2+\frac{1}{3!}\pdv[3]{u}{x}~(x_i,y_j)h^3+\frac{1}{4!}\pdv[4]{u}{x}~(\xi_i,y_j)h^4\\
        u(x_i - h, y_j) &= u(x_i, y_j) - \pdv{u}{x}~(x_i,y_j)h+\frac{1}{2!}\pdv[2]{u}{x}~(x_i,y_j)h^2-\frac{1}{3!}\pdv[3]{u}{x}~(x_i,y_j)h^3+\frac{1}{4!}\pdv[4]{u}{x}~(\eta_i,y_j)h^4
    \end{align*}
    \normalsize
    where $\xi_i \in (x_i, x_i+h)$ and $\eta_i \in (x_i-h,x_i)$. Adding the two equations together, we find
    \small
    \begin{equation*}
        u(x_i + h, y_j) + u(x_i - h, y_j) = 2u(x_i, y_j) + \pdv[2]{u}{x}~(x_i, y_j)h^2 + \frac{h^4}{4!}\left[\pdv[4]{u}{x}~(\xi_i, y_j) + \pdv[4]{u}{x}~(\eta_i, y_j)\right]
    \end{equation*}
    \normalsize
    Without loss of generality, we assume $\pdv[4]{u}{x}~(\xi_i, y_j) \leq \pdv[4]{u}{x}~(\eta_i, y_j)$. Since $\pdv[4]{u}{x}~(x,y)$ is continuous and
    \begin{equation*}
        \pdv[4]{u}{x}~(\xi_i, y_j) \leq \frac{\pdv[4]{u}{x}~(\xi_i, y_j) + \pdv[4]{u}{x}~(\eta_i, y_j)}{2} \leq \pdv[4]{u}{x}~(\eta_i, y_j),
    \end{equation*}
    by the Intermediate Value theorem, there exists $\theta_i \in (x_i - h, x_i + h)$ such that
    \begin{equation*}
        2\pdv[4]{u}{x}~(\theta_i, y_j) = \pdv[4]{u}{x}~(\xi_i, y_j) + \pdv[4]{u}{x}~(\eta_i, y_j)
    \end{equation*}
    Rearranging for the second-derivative, we arrive at the \textbf{centered-difference} formula for the second derivative:
    \begin{equation}\label{eq:x_centered_diff}
        \pdv[2]{u}{x}~(x_i, y_j) = \frac{u(x_i + h, y_j) - 2u(x_i, y_j) + u(x_i - h, y_j)}{h^2} - \frac{h^2}{12}\pdv[4]{u}{x}~(\theta_i, y_j)
    \end{equation}
    Following the same procedure, we can also come up with a centered-difference formula for the second partial derivative in the \textit{y}-direction:
    \begin{equation}\label{eq:y_centered_diff}
        \pdv[2]{u}{y}~(x_i, y_j) = \frac{u(x_i, y_j + k) - 2u(x_i, y_j) + u(x_i, y_j-k)}{k^2} - \frac{k^2}{12}\pdv[4]{u}{y}~(x_i, \phi_j)
    \end{equation}
    with $\phi \in (y_j-k,y_j+k)$. Equations~(\ref{eq:x_centered_diff})~and~(\ref{eq:y_centered_diff}) can be used in Equation~(\ref{eq:poisson}) to give
    \begin{equation}\label{eq:approx_poisson}
        \begin{split}
            &\frac{u(x_i + h, y_j) - 2u(x_i, y_j) + u(x_i - h, y_j)}{h^2} + \frac{u(x_i, y_j + k) - 2u(x_i, y_j) + u(x_i, y_j-k)}{k^2}\\
            &= f(x,y) + \frac{h^2}{12}\pdv[4]{u}{x}~(\theta_i, y_j) + \frac{k^2}{12}\pdv[4]{u}{y}~(x_i, \phi_j)
        \end{split}
    \end{equation}
    with our boundary conditions being
    \begin{equation}\label{eq:bound_conds}
        \begin{split}
            u(x_0, y_j) = g(x_0, y_j)&\text{ and }u(x_n, y_j) = g(x_n, y_j) \;\forall\;j=0,1,\dots,m\\
            u(x_i, y_0) = g(x_i, y_0)&\text{ and }u(x_i, y_m) = g(x_i, y_m) \;\forall\;i=0,1,\dots,n
        \end{split}
    \end{equation}
    This ensures that our numerical solution is correct along the borders of our rectangle \textit{R}.

    A rearrangement of Equation~(\ref{eq:approx_poisson}) yields the following:
    \begin{equation}\label{eq:pde_finite_diff}
        2\left[\left(\frac{h}{k}\right)^2 + 1\right]w_{ij} - (w_{i+1,j} + w_{i-1,j})-\left(\frac{h}{k}\right)^2 (w_{i,j+1}+w_{i,j-1}) = -h^2f(x,y)
    \end{equation}
    where $w_{ij}\approx u(x_i, y_j)$. It follows that our boundary conditions from Equation~(\ref{eq:bound_conds}) becomes
    \begin{align*}
        w_{0j} = g(x_0, y_j)&\text{ and }w_{nj} = g(x_n, y_j) \;\forall\;j=0,1,\dots,m\\
        w_{i0} = g(x_i, y_0)&\text{ and }w_{im} = g(x_i, y_m) \;\forall\;i=0,1,\dots,n
    \end{align*}
    Since we used the centered-difference formula twice, this method has an error of $O(h^2 + k^2)$.

    It is desirable to relabel the mesh points for notational convenience. In particular, we will use the following map of $\mathbb{N}^2 \to \mathbb{N}$:
    \begin{equation}\label{eq:counting}
        P(i,j) = (m-1)(i-1) + m-j
    \end{equation} 
    for $i=1,2,\dots,n-1$ and $j=1,2,\dots,m-1$. Intuitively, we are counting the mesh points from left to right and top to bottom, as seen in Figure~(\ref{fig:intuitive_grid}).
    \begin{figure}[H]
        \centering
        \includegraphics[width=0.4\linewidth]{intuitive_grid.png}
        \caption{Visualization of counting mesh points, from~\cite{burden_faires_2011}}\label{fig:intuitive_grid}
    \end{figure}
    This new labeling convention allows us to relabel our points $w_{ij} = w_l$. Finally, Equation~(\ref{eq:pde_finite_diff}) with $i=1,2,\dots,n-1$ and $j=1,2,\dots,m-1$
    gives us a system of $(n-1)(m-1)$ equations with $(n-1)(m-1)$ unknowns, which can be solved using Gaussian elimination or other techniques. A sample MATLAB implementation
    has been given in Appendix A.
    \section{Results}
    Our problem will be to find the steady state heat distribution of a 0.5 m x 0.5 m thin square metal plate. Two adjacent sides are held at $0^{\circ}$C and the remaining two sides
    increase in temperature linearly until they meet at $100^{\circ}$C. For this problem, we will use $n=m=4$.

    The solution to this distribution is given by Equation~(\ref{eq:laplace}), the Laplace equation.
    For convenience, we will take our region to be $R = \left\{(x,y)\; |\; 0 \leq x \leq 0.5,\;0\leq y \leq 0.5\right\}$.
    The boundary conditions can be realized as follows:
    \begin{equation}
        \begin{split}
            u(0,y) = 0 &\text{ and } u(x, 0) = 0\\
            u(0.5,y) = 200y &\text{ and } u(x, 0.5) = 200x
        \end{split}
    \end{equation}
    To show the process explicitly, we will use $n=m=4$. Evaluating Equation~(\ref{eq:pde_finite_diff}) for $i=1,2,3$ and $j=1,2,3$, we have

    {\singlespacing
        \begin{align*}
            4w_{11} - w_{21} - w_{01} - w_{12} - w_{10} &= 0\\
            4w_{12} - w_{22} - w_{02} - w_{13} - w_{11} &= 0\\
            4w_{13} - w_{23} - w_{03} - w_{14} - w_{12} &= 0\\
            4w_{21} - w_{31} - w_{11} - w_{22} - w_{20} &= 0\\
            4w_{22} - w_{32} - w_{12} - w_{23} - w_{21} &= 0\\
            4w_{23} - w_{33} - w_{13} - w_{24} - w_{22} &= 0\\
            2w_{31} - w_{41} - w_{21} - w_{32} - w_{40} &= 0\\
            2w_{32} - w_{42} - w_{22} - w_{33} - w_{41} &= 0\\
            2w_{33} - w_{43} - w_{23} - w_{34} - w_{42} &= 0\\
        \end{align*}
    }
    Rewriting this system using our labelling convention according to Equation~(\ref{eq:counting}), we find
    {\singlespacing
        \begin{equation}\label{eq:heat_distr}
            \begin{split}
                -w_2 + 4w_3 -w_6 &= 25\\
                -w_1 + 4w_2 - w_3 - w_5 &= 0\\
                4w_1 - w_2 - w_4 &= 0\\
                -w_3-w_5+4w_6-w_9 &= 50\\
                -w_2 - w_4 + 4w_5 - w_6 - w_8 &= 0\\
                -w_1 + 4w_4 -w_5 -w_7 &= 0\\
                -w_6 - w_8 + 4w_9 &= 150\\
                -w_5 - w_7 + 4w_8 -w_9 &= 50\\
                -w_4 + 4w_7 - w_8 &= 25
            \end{split}
        \end{equation}
    }

    where we evaluated all variables of the form $w_{0j}, w_{i0}, w_{nj}, \text{ and } w_{im}$ according to the boundary conditions and
    moved them to the right-hand side of the equation. The matrix of coefficients is nonsingnular, so we can use Gaussian elimination to solve this system, which yields
    
    {\singlespacing
        \begin{table}[H]
            \centering
            \begin{tabular}{|| c | c ||}
                \hline
                \textit{l} & $w_{l}$\\
                \hline
                \hline
                1 & 18.75\\
                2 & 12.5\\
                3 & 6.25\\
                4 & 37.5\\
                5 & 25\\
                6 & 12.5\\
                7 & 56.25\\
                8 & 37.5\\
                9 & 18.75\\
                \hline
                \hline
            \end{tabular}
            \caption{Results from solving the system in Equation~(\ref{eq:heat_distr})}
        \end{table}
    }

    Here we note that all the solutions to this PDE are exact since the solution is $u(x,y)=400xy$, so
    $\pdv[4]{u}{x} = \pdv[4]{u}{y}=0$, making our error zero.
    \section{Conclusion}
    While this method is a promising start to solving the Poisson equation, it lacks the accuracy that is required for
    some problems. In particular, this method could be improved using the iterative method involving the Gauss-Seidal algorithm, as demonstrated in \cite{burden_faires_2011},
    or an application of the classical fourth-order Runge-Kutta method could suffice. In total, this paper has demonstrated
    the feasibility to solving the Poisson equation, an elliptical PDE, and has provided a crude method of implementation as
    a proof-of-concept. 
    \newpage
    \section*{Appendix A\@: MATLAB Implementation}
    {\singlespacing
        \begin{code}
function [xdisp, ydisp, u] = myPoisson(f, g, a, b, c, d, n, m)
    % Return numerical solutions of the Poisson PDE at mesh points.
    % The PDE is of the form $u_{xx} + u_{yy} = f(x,y)$
    %
    % Params:
    % f - Function that takes in two arguments
    % g - Function that provides B.C. along x=a, x=b, y=c, y=d
    % a - Left bound of rectangle
    % b - Right bound of rectangle
    % c - Lower bound of rectangle
    % d - Upper bound of rectangle
    %
    % Returns a table of the approximated solutions of u at (x,y)
    if (a > b) || (c > d)
        error("Invalid rectangle bounds.")
    end
    if (n < 1) || (m < 1)
        error("Invalid value for n or m.")
    end
    h=(b-a)/n; k=(d-c)/m;
    x = (a:h:b)'; y = (c:k:d)';
    num_side = (n-1)*(m-1);
    xdisp = zeros(num_side, 1);
    ydisp = zeros(num_side, 1);
    u = zeros(num_side, 1);
    coeffs = zeros(num_side);
    b = zeros(num_side, 1);
    b(index(1,1:m-1)) = b(index(1,1:m-1))+g(x(1),y(2:m));
    b(index(1:n-1,1)) = b(index(1:n-1,1))+(((h/k)^2)*g(x(2:n),y(1)));
    b(index(n-1,1:m-1)) = b(index(n-1,1:m-1))+g(x(n+1), y(2:m));
    b(index(1:n-1,m-1)) = b(index(1:n-1,m-1))+(((h/k)^2)*g(x(2:n),y(m+1)));
    row = 1;
    for i=1:n-1
        for j=1:m-1
            b(row) = b(row) + -h^2 * f(x(i+1), y(j+1));
            coeffs(row,index(i,j)) = 2*((h/k)^2 + 1);
            if i ~= 1
                coeffs(row,index(i-1,j)) = -1;
            end
            if j ~= 1
                coeffs(row,index(i,j-1)) = -(h/k)^2;
            end
            if i ~= n-1
                coeffs(row,index(i+1,j)) = -1;
            end
            if j ~= m-1
                coeffs(row,index(i,j+1)) = -(h/k)^2;
            end
            row = row + 1;
        end
    end
    w = coeffs\b;
    row1 = 1;
    for i=1:n-1
        for j=1:m-1
            l = index(i,j);
            xdisp(row1) = x(i+1);
            ydisp(row1) = y(j+1);
            u(row1) = w(l);
            row1 = row1 + 1;
        end
    end

    function l = index(i, j)
        % Helper function to map mesh point indices to natural numbers
        % 1 <= i <= n-1, 1 <= j <= m-1
        l = (m-1)*(i-1)+m-j;
    end
end
        \end{code}
    }
    \newpage 
    \bibliographystyle{plain}
    \bibliography{refs}{}
\end{document}
