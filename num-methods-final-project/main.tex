\documentclass[12pt, titlepage]{article}

% For formatting
\usepackage[margin=1truein]{geometry}
\usepackage{fancyhdr}
\usepackage{setspace}
\usepackage{mathptmx}
\usepackage{cite}

% For equations
\usepackage{amsmath, amssymb}

\newcommand{\mytitle}{Numerical Solutions to Elliptical Partial Differential Equations}

\title{\mytitle}
\author{Jaden Nola}
\date{2 May 2023}

\pagestyle{fancy}
\fancyhf{}
\fancyhead[C]{\mytitle}
\fancyhead[R]{\thepage}
\renewcommand{\headrulewidth}{0pt}
\setlength{\headheight}{15pt}

\doublespacing
\begin{document}
    \maketitle
    \section*{Background} 
    According to Greenbaum~\&~Chartier~\cite{greenbaum}, a \textbf{partial differential equation} is an expression which can be written
    in the following form:
    \begin{equation}
        au_{xx} + 2bu_{xy} + cu_{yy} + du_{x} + eu_{y} + fu = g
    \end{equation}
    where \textit{a}, \textit{b}, \textit{c}, \textit{d}, \textit{e}, \textit{f}, and \textit{g} are functions of \textit{x} and \textit{y}.
    \newpage 
    \bibliographystyle{plain}
    \bibliography{refs}{}
\end{document}