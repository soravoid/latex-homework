\documentclass[12pt, titlepage]{article}

% For formatting
\usepackage[margin=1truein]{geometry}
\usepackage{fancyhdr}
\usepackage{setspace}
\usepackage{titlesec}
\usepackage{graphicx}
\usepackage{cite}

% For equations
\usepackage{amsmath, amssymb}
\usepackage{physics}
\usepackage{newtxtext, newtxmath}

\newcommand{\mytitle}{Numerical Solutions to Elliptical Partial Differential Equations}

\title{\mytitle}
\author{Jaden Nola}
\date{2 May 2023}

\pagestyle{fancy}
\fancyhf{}
\fancyhead[C]{\mytitle}
\fancyhead[R]{\thepage}
\renewcommand{\headrulewidth}{0pt}
\setlength{\headheight}{15pt}

\setlength{\parskip}{\lineskip}
\titleformat{\section}{\normalfont\fontsize{15.6}{15.6}\bfseries}{\thesection}{1em}{}
\titlespacing{\section}{0pt}{\parskip}{-\parskip}

\doublespacing
\begin{document}
    \maketitle
    \section*{Background}
    According to Greenbaum~\&~Chartier~\cite{greenbaum}, a \textbf{partial differential equation} (PDE) is an expression which can be written
    in the following form:
    \begin{equation}\label{eq:pde}
        au_{xx} + 2bu_{xy} + cu_{yy} + du_{x} + eu_{y} + fu = g
    \end{equation}
    where \textit{a}, \textit{b}, \textit{c}, \textit{d}, \textit{e}, \textit{f}, and \textit{g} are functions of \textit{x} and \textit{y}.
    From this form, we are able to further classify different types of PDEs based on a quantity known as
    the \textbf{discriminant}. It is defined as
    \begin{equation}
        D \equiv b^2 - ac
    \end{equation}
    When $D < 0$, we say that the equation is \textbf{elliptical}; if $D > 0$, then it is \textbf{hyperbolic}; 
    otherwise, if $D = 0$, then the equation is \textbf{parabolic}. As \textit{a}, \textit{b}, and \textit{c} are functions of \textit{x} and \textit{y},
    the type of PDE represented by Equation~(\ref{eq:pde}) can change types throughout the domain of the solution function.
    This paper will focus on numerical solutions to differential equations to elliptical PDEs.  
    \section*{Problem Statement}
    In this paper, we will specifically focus on the \textbf{Poisson~equation}~\cite{burden_faires_2011}, which is given by
    \begin{equation}\label{eq:poisson}
        \grad^2 u(x,y) \equiv \pdv[2]{u}{x}~(x,y) + \pdv[2]{u}{y}~(x,y) = f(x,y)
    \end{equation}
    where $\grad = \text{grad}f = \hat{\imath}\pdv{x} + \hat{\jmath}\pdv{y}$ and $\grad^2 = \grad \cdot \grad = \hat{\imath}\pdv[2]{x} + \hat{\jmath}\pdv[2]{y}$.
    In the case that Equation~(\ref{eq:poisson}) is homogenous (i.e., $f(x,y)=0$), then
    \begin{equation}\label{eq:laplace}
        \grad^2 u(x,y) = 0
    \end{equation}
    and Equation~(\ref{eq:laplace}) is called the \textbf{Laplace~equation}. Observe that since $b = 0$ and $a = c = 1$, 
    the discriminant of Equation~(\ref{eq:poisson}) is $D=-1$, thus the Poisson equation is always elliptic in any domain.

    Our goal is to approximate the solution to this PDE on the open rectangle 
    \begin{equation*}
        R = \left\{(x,y)\; |\; a < x < b,\; c < y < d\right\} \subset \mathbb{R}^2
    \end{equation*}
    Furthermore, we are also subject to boundary conditions $u(x,y) = g(x,y)\;\forall\:(x,y)\in S$, where \textit{S} is
    the boundary of the rectangle \textit{R}. If we have that \textit{f} and \textit{g} are continuous on \textit{R}, then 
    we are guarenteed a unique solution to Equation~(\ref{eq:poisson}).
    \section*{Method}
    Similar to our methods of approximating linear ordinary differential equations, we will split our rectangle into
    $n+1$ subintervals along the \textit{x}-direction and $m+1$ subintervals of equal length along the \textit{y}-direction, where $m,n \in \mathbb{N}$.
    Thus, we define $h=(b-a)/n$ and $k=(d-c)/m$, which gives $x_i = a  + ih,\; i=0,1,\dots,n$ and $y_j = c + kj, \; k=0,1,\dots,m$. These $(n+1)(m+1)$ points
    form our \textbf{grid}. In our grid, there are $(n-1)(m-1)$ points that do not lie on the border, which we will call \textbf{mesh points}.

    We assume that our solution function $u(x,y) \in C^4~(R)$, meaning that the fourth derivative is continuous on \textit{R}. In order to find an appropriate approximation to Equation~(\ref{eq:poisson}),
    we apply Taylor's theorem in the \textit{x}-direction. Carrying out this procedure, we find
    \small
    \begin{align*}
        u(x_i + h, y_j) &= u(x_i, y_j) + \pdv{u}{x}~(x_i,y_j)h+\frac{1}{2!}\pdv[2]{u}{x}~(x_i,y_j)h^2+\frac{1}{3!}\pdv[3]{u}{x}~(x_i,y_j)h^3+\frac{1}{4!}\pdv[4]{u}{x}~(\xi_i,y_j)h^4\\
        u(x_i - h, y_j) &= u(x_i, y_j) - \pdv{u}{x}~(x_i,y_j)h+\frac{1}{2!}\pdv[2]{u}{x}~(x_i,y_j)h^2-\frac{1}{3!}\pdv[3]{u}{x}~(x_i,y_j)h^3+\frac{1}{4!}\pdv[4]{u}{x}~(\eta_i,y_j)h^4
    \end{align*}
    \normalsize

    \newpage 
    \bibliographystyle{plain}
    \bibliography{refs}{}
\end{document}