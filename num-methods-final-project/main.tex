\documentclass[12pt, titlepage]{article}

% For formatting
\usepackage[margin=1truein]{geometry}
\usepackage{fancyhdr}
\usepackage{setspace}
\usepackage{titlesec}
\usepackage{mathptmx}
\usepackage{cite}

% For equations
\usepackage{amsmath, amssymb}

\newcommand{\mytitle}{Numerical Solutions to Elliptical Partial Differential Equations}

\title{\mytitle}
\author{Jaden Nola}
\date{2 May 2023}

\pagestyle{fancy}
\fancyhf{}
\fancyhead[C]{\mytitle}
\fancyhead[R]{\thepage}
\renewcommand{\headrulewidth}{0pt}
\setlength{\headheight}{15pt}

\setlength{\parskip}{\lineskip}
\titleformat{\section}{\normalfont\fontsize{12}{\lineskip}\bfseries}{\thesection}{1em}{}
\titlespacing{\section}{0pt}{\parskip}{-\parskip}

\doublespacing
\begin{document}
    \maketitle
    \section*{Background}
    According to Greenbaum~\&~Chartier~\cite{greenbaum}, a \textbf{partial differential equation} (PDE) is an expression which can be written
    in the following form:
    \begin{equation}\label{eq:pde}
        au_{xx} + 2bu_{xy} + cu_{yy} + du_{x} + eu_{y} + fu = g
    \end{equation}
    where \textit{a}, \textit{b}, \textit{c}, \textit{d}, \textit{e}, \textit{f}, and \textit{g} are functions of \textit{x} and \textit{y}.
    From this form, we are able to further classify different types of PDEs based on a quantity known as
    the \textbf{discriminant}. It is defined as
    \begin{equation}
        D \equiv b^2 - ac
    \end{equation}
    When $D < 0$, we say that the equation is \textbf{elliptical}; if $D > 0$, then it is \textbf{hyperbolic}; 
    otherwise, if $D = 0$, then the equation is \textbf{parabolic}. As \textit{a}, \textit{b}, and \textit{c} are functions of \textit{x} and \textit{y},
    the type of PDE represented by Equation~(\ref{eq:pde}) can change types throughout the domain of the solution function.
    This paper will focus on numerical solutions to differential equations to elliptical PDEs.  
    \newpage 
    \bibliographystyle{plain}
    \bibliography{refs}{}
\end{document}